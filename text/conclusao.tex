\chapter{Conclusões e Trabalhos Futuros}

Ao se o observar o problema de escalar equitativamente militares para determinados serviços, e as inúmeras falhas e fraudes geradas nesse processo, viu-se a necessidade de implementar um sistema que automatize o processo de escalação de militares e possibilitasse a auditoria em casos de suspeita de fraude. 

Desenvolvido utilizando tecnologias como PHP e Laravel framework, o sistema pode ser rapidamente desenvolvido e implementado. Devido a sua vasta documentação e inúmeras bibliotecas dispónivel, pode ainda ser expandido e facilmente mantido. O sistema desenvolvido está em sua versão beta e compreende os requisitos mínimos para a solução do problema, que é a escalação equitativa de militares e a rastreabilidade das modificações 
Com a adesão do sistema, passa-se a ter os dados centralizados e disponíveis a qualquer instante, além da rastreabilidade das modificações.

O mesmo  sistema pode ser aderido por outras áreas além do Quartel General do Exécito, mas para isso é necessário implementar as regras específicas daquele setor.

Uma outra modificação que seria benéfica e recomendável, em uma versão futura, é a separação dos dados referentes a auditoria de dados do sistema. Isso se faz necessário pois conforme o tempo passa pode se ter um volume maior de dados de auditoria do que de informações do sistema. Uma possibilidade é transferir os dados referentes a auditoria para um serviço próprio pra isso como servidores de logs, a exemplo o Graylog, que é um software livre capaz de indexar a informação permitindo buscas mais otimizadas e possui base de dados independente. Essa melhoria aumentaria ainda mais a segurança pois retiraria qualquer possibilidade de modificação dos dados por parte de pessoas que fazem manutenção do sistema.


