\chapter{Introdução}
\label{chp:introduction}
Faça aqui, uma introdução geral da área do conhecimento à qual o tema escolhido
está ligado. 

\section{Tema}
A melhor forma de determinar o tema abordado é através de hipóteses. A hipótese
consiste em uma afirmativa que você considera verdadeira e que vai provar ou
buscar provar ao longo de seu trabalho. Outra forma é delimitando o problema em
forma de uma pergunta de partida. Apresente uma visão geral do assunto que será
abordado no trabalho.

\section{Problema}
Dedique este tópico a esclarecer o que o pretende de fato com o seu esforço de
pesquisa. Problema é a questão a ser respondida pelo trabalho, que motivou a sua
realização. É uma questão que já tomou se formou em sua mente, derivada de
teorias da área pesquisada e de sua observação sobre um fenômeno.  Normalmente
se utilizam os subitens abaixo como meios de se determinar claramente os
objetivos, o que também colabora para a delimitação do escopo do trabalho. Está
estreitamente ligado ao objetivo geral, que, normalmente, consiste em encontrar
a resposta para o problema de pesquisa.  O que você viu que é um problema que
precisa de solução? É viável? Você consegue fazer? O problema é sempre uma
dificuldade, uma lacuna.\citep{risg}

\subsection{Objetivo geral}
Desenvolvimento de um sistema web para automação e controle do processo de escala de serviço de militar, possibilitando amenizar as fraudes e falhas humanas.

\subsection{Objetivos específicos}
\begin{itemize}[label=$\bullet$]
    \item Desenvolver um sistema web de escalação automática de militares.
    \item O sistema deve gerar a escala respeitando as normas impostas pelo orgão militar.
    \item O sistema deve notificar os militares que forem escalados.
    \item O sistema deve permitir que os militares corfirmem que estão cientes da escalação.
    \item O sistema deve permitir alterações na escala pelo oficial responsável.
    \item O sistema deve notificar os militares de alterações na escala.
    \item O sistema deve registrar toda e qualquer alteração na escala.
\end{itemize}

\section{Estrutura do TCC}
Neste item você vai descrever como está constituída a monografia, indicando o
que será encontrado em cada uma das sessões seguintes.

\subsection{Classificação da Pesquisa}
Neste item será apresentada a classificação da pesquisa quanto aos objetivos
(exploratória, descritiva ou explicativa); aos procedimentos (Pesquisa
bibliográfica, Pesquisa documental, Pesquisa experimental, Estudo de caso
controle, Levantamento, Estudo de caso ou Estudo de campo) e ao método de
investigação científica (qualitativa ou quantitativa).
